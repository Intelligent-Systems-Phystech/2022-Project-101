\documentclass{article}
\usepackage{arxiv}

\usepackage[utf8]{inputenc}
\usepackage[english, russian]{babel}
\usepackage[T1]{fontenc}
\usepackage{url}
\usepackage{booktabs}
\usepackage{amsfonts}
\usepackage{nicefrac}
\usepackage{microtype}
\usepackage{lipsum}
\usepackage{graphicx}
\usepackage{natbib}
\usepackage{doi}



\title{A template for the \emph{arxiv} style}

\author{ Igor ~Melnikov	\\
	Moscow Institute of Physics and Technology\\
	Dolgoprudny, Russia \\
	\texttt{melnikov.ia@phystech.edu} \\
	%% examples of more authors
	\And
	Rustem Islamov \\
	Institut Polytechnique de Paris\\
	Palaiseau, France \\
	\texttt{rustem.islamov@ip-paris.fr} \\
	%% \AND
	%% Coauthor \\
	%% Affiliation \\
	%% Address \\
	%% \texttt{email} \\
	%% \And
	%% Coauthor \\
	%% Affiliation \\
	%% Address \\
	%% \texttt{email} \\
	%% \And
	%% Coauthor \\
	%% Affiliation \\
	%% Address \\
	%% \texttt{email} \\
}
\date{}

\renewcommand{\shorttitle}{\textit{arXiv} Template}

%%% Add PDF metadata to help others organize their library
%%% Once the PDF is generated, you can check the metadata with
%%% $ pdfinfo template.pdf
\hypersetup{
pdftitle={A template for the arxiv style},
pdfsubject={q-bio.NC, q-bio.QM},
pdfauthor={David S.~Hippocampus, Elias D.~Striatum},
pdfkeywords={First keyword, Second keyword, More},
}

\begin{document}
\maketitle

\begin{abstract}
Empirical Risk Minimization problem is a common problem in machine learning methods. In the article we analyze Newton-type methods of Empirical Risk Minimization problem for some Machine Learning model using Newton-type method accessing one data point per iteration. Specifically, by applying existing sampling strategies we plan to improve stochastic second-order method presented in the paper Stochastic Newton and Cubic Newton Methods with Simple Local Linear-Quadratic Rates. We focus on sampling strategies from Parallel coordinate descent methods for big data optimization.
\end{abstract}


\end{document}
