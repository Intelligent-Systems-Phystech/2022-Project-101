\documentclass{article}
\usepackage{arxiv}

\usepackage[utf8]{inputenc}
\usepackage[english, russian]{babel}
\usepackage[T1]{fontenc}
\usepackage{url}
\usepackage{booktabs}
\usepackage{amsfonts}
\usepackage{nicefrac}
\usepackage{microtype}
\usepackage{lipsum}
\usepackage{graphicx}
\usepackage{natbib}
\usepackage{doi}



\title{A template for the \emph{arxiv} style}

\author{ Igor ~Melnikov	\\
	Moscow Institute of Physics and Technology\\
	Dolgoprudny, Russia \\
	\texttt{melnikov.ia@phystech.edu} \\
	%% examples of more authors
	\And
	Rustem Islamov \\
	Institut Polytechnique de Paris\\
	Palaiseau, France \\
	\texttt{rustem.islamov@ip-paris.fr} \\
	%% \AND
	%% Coauthor \\
	%% Affiliation \\
	%% Address \\
	%% \texttt{email} \\
	%% \And
	%% Coauthor \\
	%% Affiliation \\
	%% Address \\
	%% \texttt{email} \\
	%% \And
	%% Coauthor \\
	%% Affiliation \\
	%% Address \\
	%% \texttt{email} \\
}
\date{}

\renewcommand{\shorttitle}{\textit{arXiv} Template}

%%% Add PDF metadata to help others organize their library
%%% Once the PDF is generated, you can check the metadata with
%%% $ pdfinfo template.pdf
\hypersetup{
pdftitle={A template for the arxiv style},
pdfsubject={q-bio.NC, q-bio.QM},
pdfauthor={David S.~Hippocampus, Elias D.~Striatum},
pdfkeywords={First keyword, Second keyword, More},
}

\begin{document}
\maketitle

\begin{abstract}
We analyse Newton-type methods of Empirical Risk Minimization problem for some Machine Learning model using Newton-type method accessing one data point per iteration. Specifically, we plan to improve stochastic variant of Newton’s method. Unlike most other stochastic variants of second order methods, which require the evaluation of a large number of gradients and/or Hessians in each iteration to guarantee convergence, this methods do not have this shortcoming. We try to improve the performance of the algorithm by applying existing sampling strategies and incremental methods.
\end{abstract}


\end{document}
